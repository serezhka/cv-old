\documentclass[11pt,a4paper,russian]{moderncv}

\moderncvtheme[purple]{classic}
\usepackage[utf8]{inputenc}
\usepackage[scale=0.8]{geometry}
\usepackage[T2A]{fontenc}
\usepackage[english,russian]{babel}
\usepackage[unicode]{hyperref}
\usepackage{graphicx}
\definecolor{linkcolour}{rgb}{0,0.2,0.6}
\hypersetup{colorlinks,breaklinks,urlcolor=linkcolour, linkcolor=linkcolour}

\makeatletter
\renewcommand*{\bibliographyitemlabel}{\@biblabel{\arabic{enumiv}}}
\renewcommand{\rmdefault}{cmr}
\renewcommand{\sfdefault}{cmss}
\renewcommand{\ttdefault}{cmtt}
\renewcommand*{\makecvtitle}{
  \recomputecvlengths
  \newbox{\makecvtitledetailsbox}
  \savebox{\makecvtitledetailsbox}{
    \addressfont\color{color2}
    \begin{tabular}[t]{@{}r@{}}
      \ifthenelse{\isundefined{\@addressstreet}}{}{\makenewline\addresssymbol\@addressstreet
        \ifthenelse{\equal{\@addresscity}{}}{}{\makenewline\@addresscity}}
      \makenewline\includegraphics[width=8pt,height=8pt]{birth_logo.jpg} 08.03.1993
      \ifthenelse{\isundefined{\@mobile}}{}{\makenewline\mobilesymbol\@mobile}
      \ifthenelse{\isundefined{\@email}}{}{\makenewline\emailsymbol\emaillink{\@email}}
      \ifthenelse{\isundefined{\@homepage}}{}{\makenewline\githubsocialsymbol\httplink{\@homepage}}
      \ifthenelse{\isundefined{\@extrainfo}}{}{\makenewline\@extrainfo}
    \end{tabular}
  }
  \newbox{\makecvtitlepicturebox}
  \savebox{\makecvtitlepicturebox}{
    \ifthenelse{\isundefined{\@photo}}
    {}
    {
      \hspace*{\separatorcolumnwidth}
      \color{color1}
      \setlength{\fboxrule}{\@photoframewidth}
      \ifdim\@photoframewidth=0pt
        \setlength{\fboxsep}{0pt}\fi
  \framebox{\includegraphics[width=\@photowidth]{\@photo}}}}
  % name and title
  \newlength{\makecvtitledetailswidth}\settowidth{\makecvtitledetailswidth}{\usebox{\makecvtitledetailsbox}}
  \newlength{\makecvtitlepicturewidth}\settowidth{\makecvtitlepicturewidth}{\usebox{\makecvtitlepicturebox}}
  \ifthenelse{\lengthtest{\makecvtitlenamewidth=0pt}}% check for dummy value (equivalent to \ifdim\makecvtitlenamewidth=0pt)
    {\setlength{\makecvtitlenamewidth}{\textwidth-\makecvtitledetailswidth-\makecvtitlepicturewidth}}
    {}
  \begin{minipage}[t]{\makecvtitlenamewidth}
    \namestyle{\@firstname\ \@familyname}
    \ifthenelse{\equal{\@title}{}}{}{\\[1.25em]\titlestyle{\@title}}
  \end{minipage}
  \hfill
  % detailed information
  \llap{
    \begin{minipage}[t]{\makecvtitledetailswidth}
    \vspace*{-17pt}
    \usebox{\makecvtitledetailsbox}
    \end{minipage}}
  % optional photo (rendering)
  \begin{minipage}[t]{\makecvtitlepicturewidth}
    \vspace*{-22pt}
    \vbox to 0pt{
      \usebox{\makecvtitlepicturebox}
    }
  \end{minipage}\\[2.0em]
  % optional quote
  \ifthenelse{\isundefined{\@quote}}
    {}%
    {{\centering\begin{minipage}{\quotewidth}\centering\quotestyle{\@quote}\end{minipage}\\[2.5em]}}
  \par}% to avoid weird spacing bug at the first section if no blank line is left after \makecvtitle
\makeatother

\firstname{Сергей}
\familyname{Федоров}
\address{Санкт-Петербург, Россия}{}
\mobile{+7-921-921-0108}
\email{serezhka@xakep.ru}
\homepage{https://github.com/serezhka}
\photo[70pt][0pt]{qr-code-contact.png}
\title{CV}

\begin{document}
\maketitle

\section{Краткие сведения}
\cvline
  {}
  {Мое основое направление - разработка десктопных и веб-приложений на Java, \newline{} 
  разработка для мобильной платформы Android.}

\section{Проекты}
\cvline
  {Storage-exam}
  {Система тестирования производительности баз данных.\newline{} 
  Разработаны различные сценарии записи/чтения данных, автоматическое построение графиков производительности тестируемых баз данных.\newline{} Исходный код закрыт (Яндекс NDA)}
\cvline
  {php-ini-plugin}
  {Языковой плагин для IntelliJ IDEA 11.1. \newline{} 
  Разработаны лексер, парсер, подсветка синтаксиса, структурированный вид и контроль повторяющихся ключевых слов для php.ini (Файл конфигурации PHP).}  
\cvline
  {Micelius}
  {\url{https://github.com/serezhka/Micelius}\newline{}
  Frontend-backend service, web framework.}
\cvline
  {Optiks}
  {\url{https://github.com/Avladiev/Optiks_v2}\newline{}
  LibGDX Engine based Android game.}

\section{Опыт работы}
\cventry
  {Дек. 2010 - Июнь 2013}
  {Программист}
  {ООО "ОП "ООВО", Санкт-Петербург, Россия}
  {}{}
  {Установка, настройка и обслуживание аппаратного и программного обеспечения. 
  \newline{}Консультация пользователей и помощь при работе с программным обеспечением, локальной сетью.}
\cventry
  {Нояб. 2010 - Авг. 2011}
  {Стажер-разработчик}
  {ООО "Яндекс", Санкт-Петербург, Россия}
  {}{}
  {Разработка программного обеспечения.}
\cventry
  {Апр. 2009 - Нояб. 2010}
  {Системный администратор}
  {ООО "Филин", Санкт-Петербург, Россия}
  {}{}
  {Установка, настройка и обслуживание аппаратного и программного обеспечения. 
  \newline{}Консультация пользователей и помощь при работе с программным обеспечением, локальной сетью.}  

\section{Навыки разработки}
\subsection{Языки программирования}
\cvline
  {Intermediate}{Java}
\cvline
  {Basic}{Delphi, Visual Basic, C++}
\subsection{Фреймворки и технологии}
\cvline{Java}{Spring, Jetty, Gson}
\cvline{Android}{LibGDX, AndEngine, Universal Tween Engine}
\newpage

\section{Навыки владения компьютером}
  \cvline
  {IDE/Editors}{IntelliJ IDEA, Notepad++, Embarcadero RAD Studio (Delphi), Visual Studio}
  \cvline
  {OS}{Windows, Linux (Kali Linux, Linux Mint), MacOS X}
  \cvline
  {VCS}{Git, SVN}

\section{Образование}
  \cventry
    {}
    {\textnormal{Санкт-Петербургский национальный исследовательский университет информационных технологий, механики и оптики}{ (СПбНИУ ИТМО)}}
    {\textnormal{Естественнонаучный факульет, кафедра высшей математики}}
    {}{}{}
  \cventry
    {2010 - 2014}
    {\textnormal{Бакалавр по направлению "прикладная математика и информатика"}}
    {}{}{}{}
  \cventry
    {2014 - ...}
    {\textnormal{Магистр по направлению "математическое моделирование"}}
    {}{}{}{}
\section{Персональная информация}
  \cvline{Возраст}{22 года}
  \cvline{}{Имеются водительские права, автомобиль}

\end{document}